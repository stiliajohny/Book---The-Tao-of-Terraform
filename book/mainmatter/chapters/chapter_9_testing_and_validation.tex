\chapter{Testing and Validation}
\sloppy

\section{The Importance of Testing in Infrastructure as Code}

In the same way that software code requires testing to ensure it functions as expected, infrastructure as code (IaC) needs testing to guarantee that the configuration and automation provided by Terraform are correct. Without tests, there's always the risk that the infrastructure may be misconfigured, leading to potential downtime or security vulnerabilities.

In this chapter, we will explore various ways to test Terraform configurations, validate your infrastructure, and ensure that changes are safe before applying them. By incorporating testing and validation into your workflow, you can significantly reduce the risk of errors and improve the reliability of your infrastructure.

\section{Unit Testing Terraform Code}

Unit testing in Terraform can be challenging due to its declarative nature and the fact that it manages external resources. However, it is still possible to write tests for certain aspects of Terraform configurations. The goal of unit testing in Terraform is to ensure that the code behaves as expected under different conditions, without applying it to real infrastructure.

\subsection{Using Terraform's Built-in Validation}

Terraform has a built-in command called \texttt{terraform validate}, which checks the syntax and internal consistency of Terraform files. It does not require applying the configuration and can be used to quickly catch errors like misconfigured resource blocks or undefined variables.

Here's how to run the validation command:

\begin{lstlisting}[language=bash]
terraform validate
\end{lstlisting}

This command checks the configuration for syntax errors and missing variables. However, it does not validate whether the configuration can actually be applied successfully (i.e., it doesn't check for provider connectivity or the actual state of resources).

\subsection{Using \texttt{tflint} for Static Code Analysis}

For more advanced static analysis, \texttt{tflint} is a tool that can help detect common mistakes in Terraform configurations. It checks for issues like missing provider configurations, unquoted variables, deprecated syntax, and more.

To use \texttt{tflint}, first install it using the following command:

\begin{lstlisting}[language=bash]
brew install tflint  # For macOS
\end{lstlisting}

Next, run \texttt{tflint} on your Terraform files:

\begin{lstlisting}[language=bash]
tflint .
\end{lstlisting}

\texttt{tflint} will analyze your Terraform configurations and provide warnings and suggestions for improvement.

\section{Integration Testing with \texttt{terraform plan}}

While unit tests check the syntax and structure of your configurations, integration testing involves running Terraform commands like \texttt{terraform plan} to validate that the changes will be applied correctly to real infrastructure.

\subsection{Running \texttt{terraform plan}}

The \texttt{terraform plan} command creates an execution plan that shows what changes Terraform will apply to the infrastructure. This allows you to preview the changes before actually applying them. Running \texttt{terraform plan} is a critical step in testing changes to your infrastructure, as it helps identify potential issues that might arise when applying the configuration.

For example:

\begin{lstlisting}[language=bash]
terraform plan
\end{lstlisting}

The output will show a preview of the actions Terraform will take, such as adding, modifying, or deleting resources. By reviewing the plan before applying it, you can ensure that the changes are correct and aligned with your intentions.

\subsection{Environment-Specific Testing}

In multi-environment configurations (e.g., development, staging, production), it is important to test the infrastructure in each environment. Terraform supports multiple workspaces, which allow you to manage different environments with separate state files.

To switch to a different workspace:

\begin{lstlisting}[language=bash]
terraform workspace select staging
\end{lstlisting}

Running \texttt{terraform plan} in each workspace ensures that the configuration is valid and will apply correctly to the respective environment.

\section{Using \texttt{terraform apply} Safely}
\subsection{Using \texttt{-auto-approve} with \texttt{terraform apply}}

While \texttt{terraform plan} previews the changes, \texttt{terraform apply} actually applies the changes to your infrastructure. To avoid accidentally applying changes, always run \texttt{terraform plan} first. In automated workflows, you can use the \texttt{-auto-approve} flag to bypass the confirmation step during \texttt{terraform apply}.

Here's how to use \texttt{terraform apply} with \texttt{-auto-approve}:

\begin{lstlisting}[language=bash]
terraform apply -auto-approve
\end{lstlisting}

However, it's important to use this flag with caution, especially in production environments, as it skips the manual approval step that ensures changes are reviewed before being applied.

\section{Acceptance Testing with \texttt{terratest}}

For more advanced testing, you can use \texttt{terratest}, a Go-based testing framework for Terraform. \texttt{terratest} allows you to write automated tests that deploy real infrastructure and then validate that it is set up correctly.

To use \texttt{terratest}, write Go tests that launch your Terraform configurations and verify that the resources are properly created. For example:

\begin{lstlisting}[language=go]
package test

import (
  "testing"
  "github.com/gruntwork-io/terratest/modules/terraform"
  "github.com/gruntwork-io/terratest/modules/validator"
)

func TestTerraform(t *testing.T) {
  terraformOptions := &terraform.Options{
    TerraformDir: "../examples/terraform_example",
    Vars: map[string]interface{}{
      "ami": "ami-0c55b159cbfafe1f0",
    },
    VarFiles: []string{},
  }

  // Initialize and apply Terraform configuration
  terraform.InitAndApply(t, terraformOptions)

  // Validate the output
  output := terraform.Output(t, terraformOptions, "instance_id")
  validator.AssertNotEmpty(t, output)
}
\end{lstlisting}

In this example, the test initializes the Terraform configuration, applies it, and then checks that the output (e.g., instance ID) is not empty. \texttt{terratest} is a powerful way to automate end-to-end testing for your infrastructure.

\section{Validating Infrastructure with \texttt{infracost}}

While Terraform focuses on managing the infrastructure lifecycle, \texttt{infracost} is a tool that helps you visualize the cost implications of your infrastructure changes. By integrating \texttt{infracost} into your workflow, you can estimate the cost of your infrastructure before applying changes.

To use \texttt{infracost}, install it and run:

\begin{lstlisting}[language=bash]
infracost breakdown --path=terraform
\end{lstlisting}

This will output a cost estimate for the resources defined in your Terraform configuration, helping you make cost-conscious decisions before applying changes.

\section{Wrapping Up}

Testing and validation are critical aspects of managing infrastructure with Terraform. By using \texttt{terraform validate}, \texttt{tflint}, \texttt{terraform plan}, and other testing tools like \texttt{terratest} and \texttt{infracost}, you can ensure that your configurations are safe, accurate, and cost-effective. Incorporating testing into your Terraform workflow helps catch errors early, reduce risks, and maintain the reliability of your infrastructure.

\vspace{1em}

\textit{In the next chapter, we'll explore scaling infrastructure with Terraform. Let's go.}
